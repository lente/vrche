% chapter about the `colorization using optimization` technique
\section{Colorization using optimization}
	
	Another technique for (re)coloring images is proposed by \citet{levin2004colorization}.
	They posed that any any pixel's color should not differ too much from the colors of it's 
	neighbours.
	That is: given any pixel, the difference between it's color and the weighted sum of it's
	neighbours colors should be minimal:

	\begin{equation}
		J(U) = \sum_r \left ( U(r) - \sum_{s \in N(r)} w_rs \cdot U(s) \right )^2
	\end{equation}

	Where $U(r)$ is the color of a pixel $r = (x,y)$ and $N(r)$ is the set of neighbours of $r$ 
	(any pixel in a given neighbourhood).
	The key in this equation is the weighting function $w_rs$.
	It determines how much the color of $r$ will be like the color of $s$.
	\citet{levin2004colorization} propose the use of two possible weighting functions,
	both relying on the luminance difference between $r$ and $s$.
	
