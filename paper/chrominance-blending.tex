% comments and short explanation on the algorithm explained in "Fast
% Image and Video Colorization Using Chrominanche Blending"

\section{Fast Image and Video Colorization Using Chrominance
  Blending}

        In their paper \cite{yatziv2006fast} describe an algorithm to
        (re)colorize black and white images using scribbles thereon of color
        information\footnote{The supplied information is assumed to be correct
        however, this could also be an approximation of the correct color or
        location where this color should be.}. It is observed that in an
        original picture, the local difference in chrominance is roughly
        proportional to the local difference in luminance for each pixel.\\
        \\
        Based on this observation a simple algorithm is proposed. Every pixel
        that needs to be (re)colored will have the color of the weighed
        average of all the pixels in the scribble, where the weight is based
        on the geodesic distance to that pixel\footnote{Where each pixel is
        seen as a point $(x, y, Y)$ with $Y$ the luminance of the
        pixel}.

        \subsection{applicability}
                This algorithm in it's current form seems too
                laborious to apply in the aforementioned context. It
                is likely that stricter assumptions can be made for
                monochramatic pictures, based on which this algorithm
                can be adapted for this problem. It is interesting to
                note that this is relatively insensitive to small
                errors introduced when manually placing the scribbles,
                this due to the use of geodesic distance as a weighing factor.
        